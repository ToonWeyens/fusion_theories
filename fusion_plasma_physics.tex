\documentclass[10pt]{book}
\usepackage[T1]{fontenc}
\usepackage{ae,aecompl}
\usepackage{tipx}
\usepackage{a4wide}
\usepackage{framed}
\usepackage{ifthen}
\usepackage{setspace}
\usepackage{graphicx}
\usepackage{amsmath}
\usepackage{bbm}
\usepackage{bm}
\usepackage{array}
\usepackage{multicol}
\usepackage{amssymb}
\usepackage{color}
\usepackage{rotating}
\usepackage{comment}
\usepackage{hyperref}
\usepackage{gensymb}
\usepackage{multicol}
\usepackage{siunitx}
\usepackage{todonotes}
\usepackage{enumitem}

% the next block is the modern way of achieving 'a4wide'.
\usepackage{geometry}
\usepackage{layout}
\geometry{
  includeheadfoot,
  margin=2.54cm
}

\usepackage{subfigure}

% modern way of using tables
\usepackage{booktabs}


\graphicspath{{pictures/}}


\usepackage[dutch,USenglish]{babel}
\selectlanguage{USenglish}

\usepackage[margin=10pt,font=small,labelfont=bf]{caption}

% Use letters to label items in enumerations
\renewcommand{\theenumi}{\alph{enumi}}

% A small red pepper, used to indicate an above-average difficult question
\newcommand{\spicy}[0]{\includegraphics[width=0.02\textwidth]{pictures/spicy}}

\title{Plasma Physics \\
		--- \\
	34PPH}
\author{Toon Weyens}
\date{Version 2025/2026}

\bibliographystyle{unsrt}



\begin{document}

\maketitle


\chapter{Plasma Physics theories for Fusion}
\label{sec:fusion-plasma-physics}

This appendix contains information about plasma physics theories applied specifically to Magnetic Nuclear Fusion.
It is not intended for consumption by anyone else except the author, Toon Weyens, and is not part of the course material, though some course material will be based off it.

\section{Information per theory}

This section contains recipes for the derivation of some of the most important plasma physics theories used in Magnetic Nuclear Fusion.
They are kept to a bullet-point list, with enough detail to understand them, but without going into actual equations.

\subsection{Single-Particle Picture}
\label{subsec:single-particle}

The single-particle picture describes plasma dynamics in terms of the motion of individual charged particles in prescribed electromagnetic fields. 
It provides the microscopic foundation for understanding guiding-center motion, drifts, and confinement, which are later averaged over in kinetic and fluid theories.

\subsubsection{Numerical codes}
\todo[inline]{Add codes solving single-particle orbits, e.g. ORBIT, ASCOT, SPIRAL.}

\subsubsection{Starting point}
Newton–Lorentz equation for a charged particle:
\begin{equation}
  m \frac{d\mathbf{v}}{dt} = q \left( \mathbf{E} + \frac{\mathbf{v}\times \mathbf{B}}{c} \right), 
  \qquad
  \frac{d\mathbf{r}}{dt} = \mathbf{v}.
\end{equation}

\subsubsection{Assumptions}
\begin{itemize}
    \item \textbf{Test particle:} Electromagnetic fields are prescribed; particle does not modify them.
    \item \textbf{Collisionless:} Dynamics is determined purely by fields; collisions neglected.
    \item \textbf{Separation of scales:} Gyromotion is fast compared to drift motion and macroscopic confinement times.
\end{itemize}

\subsubsection{Steps}
\begin{enumerate}
    \item \textbf{Gyromotion:} Charged particle executes rapid Larmor rotation around magnetic field lines with frequency $\Omega_c = q B / m c$ and radius $\rho = v_\perp / \Omega_c$.
    \item \textbf{Guiding-center transformation:} Average over fast gyromotion to obtain slower drift dynamics of the guiding center.
    \item \textbf{Drift motions:} Identify drifts due to field inhomogeneities and forces, e.g.
    \begin{itemize}
        \item $\mathbf{E}\times\mathbf{B}$ drift (independent of species).
        \item Grad-$B$ and curvature drifts (depend on charge and energy).
        \item Polarization drift (time-varying electric fields).
    \end{itemize}
    \item \textbf{Adiabatic invariants:} Define invariants such as the magnetic moment $\mu = m v_\perp^2 / 2B$ that remain conserved if fields vary slowly.
\end{enumerate}

\subsubsection{Results}
\begin{itemize}
    \item Provides physical intuition for particle confinement and losses.
    \item Explains separation of time scales (gyro, bounce, drift) underlying kinetic and neoclassical theory.
    \item Yields guiding-center and drift equations used in orbit-following codes.
    \item Forms the basis for understanding trapped vs. passing particle populations in toroidal geometry.
\end{itemize}



\subsection{Test-Particle Collisional Model}

The collisional single-particle picture extends the test-particle description by including the effect of Coulomb collisions. 
Once collisions are included one must move to an $N$-body description, where all particles interact deterministically through Coulomb forces. 

Note that since solving the full $N$-body problem is impossible for a plasma ($N \gg 10^{20}$), practical models replace it by statistical descriptions (Fokker--Planck operators or Monte Carlo realizations). 
Thus, the ``collisional single-particle'' approach is best understood as a test-particle moving in prescribed fields, subject to statistically averaged collisional effects.

\subsubsection{Numerical codes}
\todo[inline]{Add orbit-following codes with collisions, e.g. ASCOT, ORBIT with collisional modules.}

\subsubsection{Starting point}
Newton–Lorentz equation (from \nameref{subsec:single-particle}) supplemented by collisional effects:
\begin{equation}
  m_a \frac{d\mathbf{v}}{dt} = q_a \left( \mathbf{E} + \frac{\mathbf{v}\times \mathbf{B}}{c} \right) + \mathbf{F}_{\text{coll},a},
\end{equation}
where $\mathbf{F}_{\text{coll},a}$ represents the cumulative effect of many small-angle Coulomb encounters with other particles.

\subsubsection{Assumptions}
\begin{itemize}
    \item \textbf{Test particle:} Electromagnetic fields are prescribed; the particle does not modify them.
    \item \textbf{Separation of scales:} Gyromotion is fast compared to drift motion and macroscopic confinement times.
    \item \textbf{Binary Coulomb collisions:} Approximated as the cumulative effect of many two-body small-angle scatterings.
    \item \textbf{Markovian approximation:} Deflections are uncorrelated; memory effects are neglected.
\end{itemize}

\subsubsection{Steps}
\begin{enumerate}
    \item \textbf{Conceptual $N$-body picture:} True deterministic collisional dynamics requires solving coupled equations for all particles. 
    This is infeasible in plasmas.
    \item \textbf{Statistical reduction:} Replace exact binary interactions with effective drag and diffusion in velocity space.  
    \item \textbf{Mathematical representations:}
    \begin{itemize}
        \item \emph{Deterministic PDE:} Fokker--Planck operator with drift and diffusion coefficients.
        \item \emph{Stochastic form:} Monte Carlo kicks applied to test particles to reproduce the same statistics.
    \end{itemize}
    \item \textbf{Collision frequencies:} Define slowing-down rate $\nu_s$, pitch-angle scattering rate $\nu_D$, and energy diffusion rate $\nu_E$.
\end{enumerate}

\subsubsection{Results}
\begin{itemize}
    \item Clarifies that true deterministic collisional dynamics is inherently $N$-body; all practical collisional models are statistical.  
    \item Explains how particles relax toward Maxwellian distributions via scattering in basic scenarios.  
    \item Provides effective collision operators for kinetic simulations (Boltzmann, Landau, Fokker--Planck).  
    \item Supplies characteristic times: collision time, slowing-down time, mean free path.  
    \item Forms the basis for orbit-following codes that include collisional scattering and runaway electron studies.  
\end{itemize}




\subsection{Kinetic theory}
\label{subsec:kinetic_theory}

Kinetic theory describes the behavior of plasmas in terms of the statistical properties of their constituent particles. It provides a framework for understanding how individual particle motions give rise to macroscopic plasma behavior, including transport phenomena and stability.

\subsubsection{Numerical codes}
\todo[inline]{Look into these codes}
Direct solvers of the full kinetic equation are rare in fusion because of extreme computational cost. 
In practice, kinetic approaches are simplified into reduced models (gyrokinetics, neoclassical, or fluid).
Examples of codes working closer to the general kinetic level include:
\begin{itemize}
    \item \textbf{Gkeyll}, \textbf{VALIS}, \textbf{VERITAS}: Vlasov--Maxwell solvers (low-dimensional or specialized geometries).
    \item \textbf{NORSE}, \textbf{CODE}, \textbf{CQL3D}: Fokker--Planck solvers, often reduced to 2D velocity space, used for RF physics and runaway electrons.
    \item \textbf{ORBIT}, \textbf{ASCOT}: Guiding-center Monte Carlo codes for fast particle dynamics.
\end{itemize}
but note that these codes typically employ additional simplifications, e.g in the form of reduced dimensionality or simplified collision models.


\subsubsection{Assumptions}
\begin{itemize}
    \item \textbf{Statistical approach:} Many particles, so we approximate the individual dynamics by smooth distribution functions.
    \item \textbf{Binary collisions:} Collisional effects are modeled as sums of pairwise interactions between species.
    \item \textbf{Quasi-neutrality:} On macroscopic scales $n_e \approx \sum_i Z_i n_i$.
\end{itemize}

\subsubsection{Steps}
\begin{enumerate}
  \item \textbf{Phase-space density (species $a$):} $f_a\left(\mathbf{r}, \mathbf{v}, t\right)$ describing density of particles of species $a$ in 6D phase space (position $\mathbf{r}$ + velocity $\mathbf{v}$), as well as time $t$
  \item \textbf{Vlasov equation (species $a$):} phase–space evolution in the collisionless limit.
  \begin{equation}
        \frac{\partial f_a}{\partial t}
        + \mathbf{v} \cdot \nabla_{\mathbf{r}} f_a
        + \frac{q_a}{m_a} \left( \mathbf{E} + \mathbf{v} \times \mathbf{B} \right) \cdot \nabla_{\mathbf{v}} f_a
        = 0
  \end{equation}

  \item \textbf{Boltzmann equation (species $a$):} include general binary-collision operators with all other species $b$.
  It is written generally for any two-body potential.
  In practice it is most useful for short-range, hard-sphere–like interactions, because then the cross section is well defined and large-angle scattering dominates.
  \begin{equation}
    \frac{\partial f_a}{\partial t} + \dots = \sum_b C_{ab}^{\text{Boltz}}[f_a,f_b] \quad ,
  \end{equation}
  with the Boltzmann collision operator $C_{ab}^{\text{Boltz}}[f_a,f_b]$ treating each collision as a discrete jump in velocity, removing and adding to populations at these velocities:
    \begin{equation}
        C^\text{Boltz}_{ab}[f_a,f_b](\mathbf{v}) =
        \int_{\mathbb{R}^3}\! d^3\mathbf{v}_2 \int d\Omega\;
        g\, \frac{d\sigma_{ab}}{d\Omega}(g,\Omega)\,
        \Big[ f_a(\mathbf{v}_a^\star)\,f_b(\mathbf{v}_b^\star) - f_a(\mathbf{v})\,f_b(\mathbf{v}_2) \Big] \quad ,
    \end{equation}
    where $\mathbf{v}^\star$ is the pre-collision velocity that leads to post-collision velocity $\mathbf{v}$ (and similarly for $\mathbf{v}_2^\star$), $g = |\mathbf{v} - \mathbf{v}_2|$ is the relative speed, and $d\sigma_{ab}/d\Omega$ is the differential cross section for scattering from species $a$ and $b$.

  \item \textbf{Landau operator:} Coulomb–specific small–angle limit of Boltzmann for species $a$ colliding with $b$. 
  \begin{equation}
    C_{ab}^{\text{Landau}}[f_a,f_b]
  \end{equation}
  Note that the cross section for Boltzmann is very forward-peaked: almost all encounters are tiny deflections.
  Therefore, if you insert the Coulomb cross section into the Boltzmann collision integral, the integral diverges (because of the infinite range of $1/r$).
  To make this work mathematically Landau treated the Coulomb case by expanding the Boltzmann operator in the limit of small deflection angles and summing the cumulative effect of many such encounters.
  That approximation turns the discrete “jump” operator that is typical for Boltzmann into a Fokker–Planck (drift–diffusion) operator in velocity space, called the Landau collision operator.


  \item \textbf{Fokker–Planck equation:} general drift–diffusion form of kinetic equation.
  In plasma physics, the Landau operator provides the explicit Coulomb version.
  \begin{equation}
    \frac{\partial f_a}{\partial t} + \dots = \sum_b C_{ab}^{\text{FP}}[f_a,f_b]
  \end{equation}
\end{enumerate}

\subsubsection{Results}
Quantities and insights from kinetic theory, forming the foundation for higher-level models:
\begin{itemize}
    \item \textbf{Primary:} Distribution functions $f_a(\mathbf{r},\mathbf{v},t)$ for each species; explicit collisional operators $C_{ab}$ (Boltzmann, Landau, Fokker--Planck).
    \item \textbf{Derived:} Collision frequencies, mean free paths, slowing-down times, Spitzer resistivity, classical transport coefficients.
    \item \textbf{Applications:} Basis for fluid models (via moments), neoclassical transport (via drift-kinetic reductions), and validation of turbulence/transport simulations.
\end{itemize}



\subsection{Neoclassical Transport}

\subsubsection{Numerical codes}
\todo[inline]{Add info about these codes}
\begin{itemize}
    \item Astra
    \item TRANSP
    \item JINTRAC
\end{itemize}

\subsubsection{Starting point}
The Fokker--Planck equation (See \nameref{subsec:kinetic_theory})

\subsubsection{Assumptions}
\begin{itemize}
    \item \textbf{Flux-surface containment:} To zeroth order, particles and energy are confined to magnetic flux surfaces. 
        Radial fluxes across flux surfaces are assumed to be small and appear only at first order (due to drifts, collisions, and electric fields).  
    \item \textbf{Small gyroradius:} Ordering $\rho / a \ll 1$ allows guiding-center reduction and drift-kinetic treatment.
    \item \textbf{High gyrofrequency:} Drift and collision frequencies are much smaller than the cyclotron frequency, $\nu, \omega_d \ll \Omega_c$, so gyro-motion can be averaged out.
    \item \textbf{Linearization:} Distribution function split into $f_a = f_{a,0} + f_{a,1}$ with $f_{a,1} \ll f_{a,0}$, enabling perturbative treatment around Maxwellian $f_{a,0}$.
\end{itemize}

\subsubsection{Steps}
\begin{enumerate}
  \item \textbf{Kinetic theory:} Start from the Fokker--Planck equation for the distribution function $f(\mathbf{r},\mathbf{v},t)$, including collisions.
    \begin{equation}
        \frac{\partial f_a}{\partial t}
        + \mathbf{v} \cdot \nabla_{\mathbf{r}} f_a
        + \frac{q_a}{m_a} \left( \mathbf{E} + \mathbf{v} \times \mathbf{B} \right) \cdot \nabla_{\mathbf{v}} f_a
        = \sum_b C_{ab}[f_a, f_b],
    \end{equation}

  \item \textbf{Drift-kinetic equation:} In strong magnetic field $\mathbf{B}$, apply guiding-center reduction. Averaging over the gyro-angle yields reduced equations for the guiding-center coordinates $(\mathbf{R}, v_{\parallel}, \mu)$ instead of $(\mathbf{r}, \mathbf{v})$.
  \begin{itemize}
    \item $\mathbf{R}$: guiding center position, often in flux coordinates.
    \item $v_{\parallel}$: parallel velocity.
    \item $\mu = m v_{\perp}^2 / 2B$: magnetic moment, encoding perpendicular kinetic energy.
  \end{itemize}

  \item \textbf{Linearization:} Linearize drift-kinetic equation around Maxwellian equilibrium $f_a = f_{a,0} + f_{a,1}$ with $f_{a,0}$ Maxwellian and $f_{a,1} \ll f_{a,0}$.
  \begin{itemize}
    \item $f_{a,0}$: Maxwellian, written in terms of $n_a$, $T_a$, and optionally $u_{\parallel a}$ for consistency with nonzero parallel bulk flow (e.g. Ohmic current drive):
        \begin{equation}
        f_{a,0}(\mathbf{v}) = n_a \left( \frac{m_a}{2\pi T_a} \right)^{3/2}
        \exp\!\left[
        - \frac{m_a}{2T_a} \left( (v_{\parallel} - u_{\parallel a})^2 + v_{\perp}^2 \right)
        \right],
        \end{equation}

    \item $f_{a,1}$: introduces anisotropy from $\nabla n$, $\nabla T$, $E_r$, magnetic drifts. Collisionality determines trapped vs. passing particle behavior (banana, plateau, Pfirsch--Schl\"uter regimes).
  \end{itemize}

  \item \textbf{Solve for $f_{a,1}$:} Encodes how gradients, geometry, and fields perturb the local Maxwellian, providing fluxes.

  \item \textbf{Velocity-space moments:}
  \begin{itemize}
    \item For $f_{a,0}$: trivial, yielding $n_a$, $\mathbf{u}_a$, $T_a$ or $p_a$.
    \item For $f_{a,1}$: leads to radial fluxes:
    \begin{align}
        \Gamma_{\psi a} &\sim - D_a \nabla n_a + \dots, \\
        q_{\psi a} &\sim - \chi_a \nabla T_a + \dots, \\
        j_{\parallel a} &\neq 0 \quad \text{even if } E_\parallel = 0 \quad (\text{bootstrap current}).
    \end{align} 
  \end{itemize}

  \item \textbf{Constraints:} Determine the radial electric field $E_r$ by imposing ambipolarity (no net current across flux surfaces, which is already true in axisymmetry)
  \begin{equation}
    \sum_a Z_a e \, \Gamma_{\psi a}(E_r) = 0,
  \end{equation}
  relate densities by assuming quasi-neutrality (Debye shielding is very fast)
  \begin{equation}
    \sum_a Z_a n_a \approx 0,
  \end{equation}
  and parallel flows for each species $u_{\parallel a}$ by using parallel momentum balance (also known as force balance).

  \item \textbf{Closure:} Substitute flux expressions into profile evolution equations $n_a(\mathbf{R},t)$, $T_a(\mathbf{R},t)$, etc.

  \item \textbf{Flux-surface averaging:} Average over flux surfaces, obtaining transport equations as functions of flux coordinate $\psi$ and time $t$.

  \item \textbf{Transport equations:} Evolve profiles with sources/sinks toward steady state or in time.

\end{enumerate}

\subsubsection{Results}
Quantities and insights used in higher-level models (e.g., MHD):
  \begin{itemize}
    \item \textbf{Primary:} Particle flux $\Gamma_{\psi a}$, heat flux $q_{\psi a}$, parallel current $j_{\parallel a}$. 
    \item \textbf{Derived:} Transport coefficients ($D_a, \chi_a$), bootstrap current, collisionality-regime scalings. 
  \end{itemize}



\subsection{General MHD}

Magnetohydrodynamics (MHD) describes a plasma as a conducting fluid interacting with electromagnetic fields. 
It is obtained by taking velocity moments of the kinetic equation and combining them with Maxwell’s equations. 
This subsection presents the general framework; specific closures (ideal, resistive, two-fluid, Braginskii) are introduced in later subsections.

\subsubsection{Numerical codes}
Have a look at the next subsections instead, as they contain more specific MHD models with their own codes.

\subsubsection{Starting point}
Fokker–Planck equation for the distribution function $f_a(\mathbf{r},\mathbf{v},t)$ of species $a$ (see \nameref{subsec:kinetic_theory})

\subsubsection{Assumptions}
\begin{itemize}
    \item \textbf{Fluid moment hierarchy:} Use velocity-space moments for each species. 
    Typically truncate after (0) density, (1) velocity, and (2) pressure/temperature, unless higher-order closures are needed.
    \item \textbf{Quasi-neutrality:} Net charge density is small on macroscopic scales, $\sum_a Z_a n_a \approx 0$.
    \item \textbf{Macroscopic ordering:} Valid on scales much larger than gyroradius and times longer than gyroperiods.
    \item \textbf{Low-frequency:} Characteristic velocities $\ll c$, displacement current neglected in Ampère’s law.
\end{itemize}

\subsubsection{Steps}
\begin{enumerate}
    \item \textbf{Kinetic theory:} Start from the Fokker--Planck equation for the distribution function $f(\mathbf{r},\mathbf{v},t)$, including collisions.
    \begin{equation}
        \frac{\partial f_a}{\partial t}
        + \mathbf{v} \cdot \nabla_{\mathbf{r}} f_a
        + \frac{q_a}{m_a} \left( \mathbf{E} + \mathbf{v} \times \mathbf{B} \right) \cdot \nabla_{\mathbf{v}} f_a
        = \sum_b C_{ab}[f_a, f_b],
    \end{equation}

    \item \textbf{Take velocity moments per species:} This leads to equations for density $n_a$, bulk velocity $\mathbf{u}_a$, and temperature $T_a$ (or pressure $p_a = n_a T_a$) for each species $a$.
    Note that the equation resulting from a moment (i) depends on quantities from the next higher moment (i+1), leading to a hierarchy that needs closure.

    \item \textbf{Introduce Maxwell’s equations:} Needed to evolve $\mathbf{B}$ and to replace $\mathbf{E}$ and $\mathbf{J}$ in terms of fluid variables, ensuring self-consistent coupling of fields and plasma.
    
    \item \textbf{Coupling:} The species equations are linked through $\mathbf{E}, \mathbf{B}, \mathbf{J}$ and collisional friction terms $\mathbf{R}_a$ (momentum and energy exchange).

    \item \textbf{Closure:} Truncate the hierarchy by modeling the pressure tensor $\mathbf{P}_a$, heat flux $\mathbf{q}_a$, and Ohm’s law. Different closures define specific MHD models:
    \begin{itemize}
        \item Ideal MHD: isotropic pressure, infinite conductivity.
        \item Resistive MHD: finite resistivity.
        \item Two-fluid / extended MHD: keep Hall term, electron pressure gradient, FLR corrections.
        \item Braginskii MHD: collisional transport closures for viscosity and heat flux.
    \end{itemize}
\end{enumerate}

\subsubsection{Results}
\begin{itemize}
    \item General MHD gives a multi-fluid framework: continuity, momentum, and energy equations for each species coupled via Maxwell’s equations.
    \item To reduce complexity, closures are required for $\mathbf{P}_a$, $\mathbf{q}_a$, and the collisional terms $\mathbf{R}_a$.
    \item Special cases arise depending on modeling choices, e.g.\ ideal vs.\ resistive Ohm's law, single-fluid vs.\ multi-fluid, and collisional vs.\ collisionless closures.
    In general, MHD models can be distinguished along several orthogonal dimensions:
    \begin{itemize}
        \item \textbf{Number of fluids:} single-fluid (bulk plasma), two-fluid (ions and electrons), or multi-fluid (separate ion species plus electrons).
        \item \textbf{Ohm's law closure:} ideal (infinite conductivity), resistive (finite $\eta$), or extended (Hall term, electron pressure gradient, inertia).
        \item \textbf{Transport closure:} scalar pressure, Braginskii (anisotropic collisional transport), or alternative closures (neoclassical, anomalous/turbulent).
        \item \textbf{Collisionality regime:} collisionless vs.\ collisional vs.\ hybrid models (adding anomalous transport on top of collisional).
        \item \textbf{Geometry/ordering:} full toroidal vs.\ reduced MHD, slab/cylindrical approximations, strong guide-field limits.
    \end{itemize}
    \item Applications: equilibrium, stability, reconnection, transport modeling.
\end{itemize}

\end{document}